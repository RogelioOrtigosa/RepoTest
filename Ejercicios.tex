\documentclass[preprint,12pt]{elsarticle}
%%%%%%%%%%%%%%%%%%%%%%%%%%%%%%%%%%%%%%%%%%%%%%%%%%%%%%%
%%%%%%%%%%%%%%%%%%%%%%%%%%%%%%%%%%%%%%%%%%%%%%%%%%%%%%%
% Load Packages Rogelio
%%%%%%%%%%%%%%%%%%%%%%%%%%%%%%%%%%%%%%%%%%%%%%%%%%%%%%%
%%%%%%%%%%%%%%%%%%%%%%%%%%%%%%%%%%%%%%%%%%%%%%%%%%%%%%%

\usepackage{geometry}
 \geometry{
 a4paper,
 total={170mm,257mm},
 left=18mm,
 right = 18mm,
 top=20mm,
 }


%\usepackage[T1]{fontenc}
%\usepackage[T1]{fontenc}
%\usepackage[utf8]{inputenc}
%\usepackage{textcomp}
\usepackage{eurosym}

\usepackage{framed}
\usepackage{titlesec}
\usepackage{amssymb}
\usepackage{moreverb}
\usepackage{amssymb,bm,upgreek,amsbsy,graphicx,color,amsmath,amsfonts,framed}
\usepackage{mathrsfs}
\usepackage{latexsym}
\usepackage{pseudocode}
\usepackage{epsfig,epsf,rotating}
\usepackage{subfig}
\usepackage{epstopdf}
\usepackage{float}
\usepackage{bbm}
\usepackage{booktabs}
\usepackage{tikz}
\usepackage{blindtext}
\usepackage{tcolorbox}
\newcommand{\ra}[1]{\renewcommand{\arraystretch}{#1}}
%\newcommand{\Cross}{\mathbin{\tikz [x=1.4ex,y=1.4ex,line width=.25ex] \draw (0.1,0.1) -- (0.9,0.9) (0.1,0.9) -- (0.9,0.1);}}
\biboptions{sort&compress}
\newcommand{\vect}[1]{\boldsymbol{#1}}
\newcommand*{\Union}{\bigcup}

\input{stmary} % include jump condition
\graphicspath{{../}}

\usepackage{arydshln}
\usepackage{tabularx}
\usepackage{array}
\usepackage{colortbl}
\usepackage{tcolorbox}
\tcbuselibrary{skins}

\usepackage{tikz,tcolorbox,array,tabularx,colortbl}

\biboptions{sort&compress}

\input{stmary} % include jump condition
\usepackage[colorlinks,bookmarksopen,bookmarksnumbered,citecolor=red,urlcolor=red]{hyperref}
\usepackage{color}
\newcommand{\Blue}[1]{\textcolor[rgb]{0.00,0.00,0.00}{#1}}
\newcommand{\Red}[1]{\textcolor[rgb]{1.00,0.00,0.00}{#1}}


\tcbuselibrary{skins}
\usetikzlibrary{shapes,arrows,matrix,positioning}
% MAKE TABLE
\newcolumntype{Y}{>{\raggedleft\arraybackslash}X}% see tabularx
\tcbset{enhanced,colback=blue!3,colframe=red!40,colbacktitle=blue!4,
coltitle=black,center title, width=2.5cm}

\usepackage{fancyhdr}
\pagestyle{fancy}
\fancyhf{}
\setlength\headheight{80.0pt}
\addtolength{\textheight}{-80.0pt}
\rhead{\includegraphics[height=0.1\textheight]{images/upct_logo}}
\lhead{\includegraphics[height=0.1\textheight]{images/etsii_logo}}
\rfoot{P\'agina \thepage}

\graphicspath{{./pictures_paper/}}


\journal{Computer Methods in Applied Mechanics and Engineering}

% Matlab2Tikz
\usepackage{tikzscale,pgfplots}
\pgfplotsset{compat=1.13}
\newcommand\figurescale{1} 
\newlength\figH
\newlength\figW

%%%%%%%%%%%%%%%%%%%%%%%%%%%%%%%%%%%%%%%%%%%%%%%%%%%%%%%
%%%%%%%%%%%%%%%%%%%%%%%%%%%%%%%%%%%%%%%%%%%%%%%%%%%%%%%
% Load Packages Alex, Marlon
%%%%%%%%%%%%%%%%%%%%%%%%%%%%%%%%%%%%%%%%%%%%%%%%%%%%%%%
%%%%%%%%%%%%%%%%%%%%%%%%%%%%%%%%%%%%%%%%%%%%%%%%%%%%%%%

%%% Packages
\usepackage{psfrag,epsfig}      
\usepackage{longtable,multirow}
	
	% Definitions
      \renewcommand{\vec}[1]{\ensuremath{\mbox{\boldmath $\mathrm{#1}$}}}
     
   	% Operators
         \renewcommand{\d}[1][]{\ensuremath{\mathrm{d}#1}}     
      \newcommand{\dx}{\ensuremath{\operatorname{dx}}}
      \newcommand{\dX}{\ensuremath{\operatorname{dX}}}
      \newcommand{\dV}{\ensuremath{\operatorname{dV}}}
      \newcommand{\dA}{\ensuremath{\operatorname{dA}}}
      \newcommand{\da}{\ensuremath{\operatorname{da}}}
      \newcommand{\dv}{\ensuremath{\operatorname{dv}}}
      \newcommand{\transp}{\ensuremath{^{\scriptstyle \mathrm{T}}}}
      \newcommand{\h}{\ensuremath{^{\scriptstyle \mathrm{h}}}}
      \renewcommand{\det}{\ensuremath{^{\scriptstyle \mathrm{det}}}}
      \newcommand{\el}{\ensuremath{^{\scriptstyle \mathrm{e}}}}
      \newcommand{\ti}{\ensuremath{_{\scriptstyle \mathrm{t}}}}
      \renewcommand{\div}{\ensuremath{\operatorname{div}}}
      \newcommand{\grad}{\ensuremath{\operatorname{grad}}}
      \newcommand{\tr}{\ensuremath{\operatorname{tr}}}
      \newcommand{\Grad}{\ensuremath{\operatorname{Grad}}}
      \newcommand{\invT}{\ensuremath{^{\scriptstyle \mathrm{-T}}}}
      

  
 % All packages included


\begin{document}


\vspace{15cm}

\begin{center}
\underline{\textbf{EJERCICIOS INTEGRALES DOBLES}}
\end{center}


A continuaci\'on se plantean los siguientes problemas. Tened en cuenta que:

\begin{itemize}
	\item [\textbf{a)}] Confiamos en que vais a trabajar individualmente los ejercicios, de hecho, pensad que son perfectos para comprobar que hab\'eis entendido todo.
	\item [\textbf{b)}]  No obstante, os aconsejamos que contact\'eis para cualquier duda al profesor Rogelio Ortigosa ({\textcolor[rgb]{0,0,1}{rogelio.ortigosa@upct.es}}) en el horario: \textbf{jueves de 15 a 20 horas}. El profesor os proporcionar\'a un link para atender las tutor\'ias a trav\'es de la aplicaci\'on \textbf{teams}.
	
	\item [\textbf{c)}] La {fecha m\'axima de entrega} ser\'a el \textbf{13 de Junio a las 23:55}.
	
	\item [\textbf{d)}] S\'olamente debe subirse un \'unico documento pdf. Recordad que pod\'eis utilizar aplicaciones en el m\'ovil para escanear las fotos que tom\'eis sobre papel.


\end{itemize}


\vspace{2mm}

\begin{enumerate}
	\item $\int_{\Omega}x y^{3/2}\,dxdy$, donde $\Omega$ es el recinto: $\Omega=\{(x,y)\in\mathbb{R}^2\,\vert\, 1\leq x^2+y^2\leq 4,\,y\geq 0,\,y\geq x\}$
	
	\vspace{4mm}
	
	\item $\int_{\Omega}x y\,dxdy$, donde $\Omega$ es el recinto limitado por las curvas $y=x^2-3x+2$ y $y=x-3/2$, es decir, $\Omega=\{(x,y)\in\mathbb{R}^2\,\vert\, y\geq x^2-3x+2,\,y\leq x-3/2\}$
	
	
	\vspace{4mm}
	
	\item $\int_{\Omega}x+y\,dxdy$, donde $\Omega$ es el reciento limitado por las curvas $y=-x^2+1$, $y=-2x+1$ y $x=1$, es decir, $\Omega=\{(x,y)\in\mathbb{R}^2\,\vert\, y\leq -x^2+1,\,y\geq -2x+1,\,x\leq 1\}$
	

\end{enumerate}


%%%%%%%%%%%%%%%%%%%%%%%%%%%%%%%%%%%%%%%%%%%%%%%%%%




%\bibliographystyle{model1-num-names}
\bibliographystyle{plain}
\bibliography{biblio}


\end{document}

